\documentclass{beamer}



\usepackage[utf8]{inputenc} 
\usepackage[T1]{fontenc}
\usepackage[frenchb]{babel}
\usepackage{setspace}
\usepackage{color}
\usepackage{listings}

\usetheme{Warsaw}
\title{Modern C++}

\author{Marcin Serwach}
%\institute{Personal}
\date{Łódź, 8.12.2016}


\lstdefinestyle{customcpp}{
  belowcaptionskip=1\baselineskip,
  breaklines=true,
  xleftmargin=\parindent,
  language=C++,
  showstringspaces=false,
  basicstyle=\footnotesize\ttfamily,
  keywordstyle=\bfseries\color{blue},
  commentstyle=\itshape\color{green!40!black},
  identifierstyle=\color{black},
  stringstyle=\color{red},
   numbers=left,                   
  numbersep=5pt,                   
  numberstyle=\tiny\color{gray},
  keepspaces=true, 
}

\begin{document}
\lstset{language=C++}

\maketitle

\begin{frame}{C++11}
\uncover<1->{
\begin{itemize}
\item Move semantic
\item Lambda
\item Automatic type deduction
\item Variadic templates
\item Enum class
\item Suffix
\item Const expression
\item initializer\_list
\item Attributes: \textit{final}, \textit{override}, \textit{default}, \textit{delete}
\item Foreach
\item Attribute noexcept
\item Brackets: \{\}
\item Calling constructor from constructor
\item nullptr
\end{itemize}
}
\end{frame}

\begin{frame}{C++11 STL}
\uncover<1->{
\begin{itemize}
\item Thread
\item Smart pointers
\item Chrono
\item Random engines/distrubution
\item Type traits
\item Regex
\item std::array
\item Tuple
\item System errors
\item Ratio
\item \textit{std::bind}, \textit{std::function}
\end{itemize}
}
\end{frame}


\begin{frame}[fragile]{Move semantic}
\begin{center}
{\Huge \textbf{Move semantic}}
\end{center}
\end{frame}

\begin{frame}[fragile] {R-value and L-value reference}
%\uncover<1->{
\textbf{L-value reference}\\
Alias name of object. \\
\begin{lstlisting}[style=customcpp]
Class obj;
Class& lref = obj;
\end{lstlisting}

\textbf{R-value reference}\\
Reference to a temporary object, that will be destroyed in near future. \\
\begin{lstlisting}[style=customcpp]
Class obj1, obj2;
Class&& rref1 = Class{}; // r-value as returned valued
Class&& rref2 = obj1 + obj2; // as above
Class&& rref3 = std::move(obj1); // calling std::move
void fun(Class&& aArg); // r-value as an argument of a method
/* ... */
fun(Class{}); // creating temporary object
\end{lstlisting}

%}

\end{frame}



\begin{frame}[fragile] {R/L-value reference and function overloading}
\begin{lstlisting}[style=customcpp]
void fun(Class& aArg){
	std::cout<<"L-value\n";
}
void fun(Class&& aArg){
	std::cout<<"R-value\n";
}
int main() {
	Class a;
	fun(a); // Output: L-value
	fun(Class{}); // Output: R-value
	return 0;
}
\end{lstlisting}

\end{frame}


\begin{frame}[fragile] {Universal reference}
\textbf{Universal reference}\\
\textbf{Only} for templates. Notation the same as R-value reference.
\begin{lstlisting}[style=customcpp]
template<typename T>
void fun(T&& aArg){
}
\end{lstlisting}
The \textit{fun} allows passing L-value and R-value reference.
\end{frame}

\begin{frame}[fragile] {Universal reference}
\begin{lstlisting}[style=customcpp]
template<typename T>
void fun(T&& aArg){
	std::cout<<"fun\n";
}

int main() {
	int a = 2, b = 1;
	int& lvalue = a;
	int&& rvalue = std::move(b);
	fun(lvalue);
	fun(rvalue);
	return 0;
}

\end{lstlisting}
\end{frame}


\begin{frame}[fragile] {\textit{std::move} and \textit{std::forward}}
\begin{lstlisting}[style=customcpp]
int a = 2;
int&& rvalue1 = std::move(a);
\end{lstlisting}
The \textit{std::move} \textbf{doesn't} move anything. It is a \textbf{unconditional} cast to r-value reference. Moved object cannot be used. \\
\begin{lstlisting}[style=customcpp]
template<typename T>
void fun(T&& aArg){
	anotherFun(std::forward<T>(aArg));
}
\end{lstlisting}
The \textit{std::forward} \textbf{doesn't} forward/move anything. It is a \textbf{conditional} cast to r-value reference. If a given argument of \textit{forward} is r-value, it casts to r-value. Otherwise \textit{forward} casts to l-value reference.

\end{frame}

\begin{frame}[fragile] {\textit{std::forward} - example without \textit{std::forward}}
\begin{lstlisting}[style=customcpp]
void anotherFun(Class& aArg){
	std::cout<<"L-value\n";
}
void anotherFun(Class&& aArg){
	std::cout<<"R-value\n";
}
template<typename T>
void fun(T&& aArg){
	anotherFun(aArg);
}
// ...
Class obj1;
fun(obj1); // Output: L-value
fun(Class{}); // Output: L-value
\end{lstlisting}
\end{frame}


\begin{frame}[fragile] {\textit{std::forward} - example with \textit{std::forward}}
\begin{lstlisting}[style=customcpp]
void anotherFun(Class& aArg){
	std::cout<<"L-value\n";
}
void anotherFun(Class&& aArg){
	std::cout<<"R-value\n";
}
template<typename T>
void fun(T&& aArg){
	anotherFun(std::forward<T>(aArg));
}
// ...
Class obj1;
fun(obj1); // Output: L-value
fun(Class{}); // Output: R-value
\end{lstlisting}
\end{frame}

\begin{frame}[fragile] {\textit{std::forward} - example explanation}
\begin{lstlisting}[style=customcpp]
void anotherFun(Class& aArg){
	std::cout<<"L-value\n";
}
void anotherFun(Class&& aArg){
	std::cout<<"R-value\n";
}
// ..
Class&& a = Class{};
anotherFun(a); // Output: L-value
\end{lstlisting}
R-value reference is passing to a method as L-value reference. To pass it as R-value reference you have to use \textit{std::move}.

\end{frame}

\begin{frame}[fragile] {Why it is so important?}
\begin{lstlisting}[style=customcpp]
void fun(Class&& aArg){
	std::cout<<"fun\n";
}
\end{lstlisting}
In method \textit{fun} the argument is passing by r-value reference, what means that it doesn't have to be valid outside of it, because it is a temporary object and it will be destroyed while leaving the method.  Therefore you can take everything from the argument, i. e.: replace pointers:
\begin{lstlisting}[style=customcpp]
void fun(Class&& aArg){
	this->iPointer = aArg.iPointer;
	aArg.iPointer = nullptr;
}
\end{lstlisting}
The argument \textbf{can't} be \textit{const}.
\end{frame}


\begin{frame}[fragile] {Move constructor and move assignment operator}
A move constructor is similar to a copy constructor with a small difference - an argument is passing by nonconst r-value reference.
\begin{lstlisting}[style=customcpp]
Class(Class&& aArg) : iPointer{aArg.iPointer} {
  aArg.iPointer = nullptr;
}
\end{lstlisting}

A move assignment operator it is a R-value version of an assignment operator.
\begin{lstlisting}[style=customcpp]
Class& operator=(Class&& aArg) {
  iPointer = aArg.iPointer;
  aArg.iPointer = nullptr;
  return *this;
}
\end{lstlisting}
\end{frame}


\begin{frame}[fragile]{Rule of 3; now it is Rule of 5}
 The Rule of 3 in C++11 is extended to the Rule of 5, because of R-value related methods. Therefore the following method should be implemented: 
 \begin{itemize}
 \item Copy constructor
 \item Move constructor
 \item Assignment operator
 \item Move assignment operator 
 \item Virtual destructor
 \end{itemize}
\end{frame}


\begin{frame}[fragile]{What about setters?}
\begin{lstlisting}[style=customcpp]
void Class::setString(std::string aArg) {
  iString = std::move(aArg);
}
\end{lstlisting}
\textbf{vs}
\begin{lstlisting}[style=customcpp]
void Class::setString(const std::string& aArg){
  iString = aArg;
}
void Class::setString(std::string&& aArg){
  iString = std::move(aArg);
}
\end{lstlisting}
\end{frame}


\begin{frame}[fragile]{What about setters? }
\begin{figure}[!htb]
    \centering
    \begin{minipage}{.5\textwidth}
        \centering
\textbf{By value}
    \end{minipage}%
    \begin{minipage}{0.5\textwidth}
        \centering
\textbf{By reference}
    \end{minipage}
    \end{figure}
    
\begin{lstlisting}[style=customcpp]
cl.setString("abc");
\end{lstlisting}

\begin{figure}[!htb]
    \centering
    \begin{minipage}{.5\textwidth}
        \centering
\begin{itemize}
\item constructor
\item move assignment operator
\end{itemize}
    \end{minipage}%
    \begin{minipage}{0.5\textwidth}
        \centering
\begin{itemize}
\item constructor
\item move assignment operator
\end{itemize}
    \end{minipage}
\end{figure}
    
\begin{lstlisting}[style=customcpp]
std::string myString{"abc"};
cl.setString(myString);
\end{lstlisting}
\begin{figure}[!htb]
    \centering
    \begin{minipage}{.5\textwidth}
        \centering
\begin{itemize}
\item copy constructor
\item move assignment operator
\end{itemize}
    \end{minipage}%
    \begin{minipage}{0.5\textwidth}
        \centering
\begin{itemize}
\item  assignment operator
\end{itemize}
    \end{minipage}
\end{figure}
\end{frame}


\begin{frame}[fragile]{What about setters? }

\begin{lstlisting}[style=customcpp]
std::string myString{"abc"};
cl.setString(std::move(myString));
\end{lstlisting}
\begin{figure}[!htb]
    \centering
    \begin{minipage}{.5\textwidth}
        \centering
\textbf{By value}
\begin{itemize}
\item move constructor
\item move assignment operator
\end{itemize}
    \end{minipage}%
    \begin{minipage}{0.5\textwidth}
        \centering
\textbf{By reference}
\begin{itemize}
\item move assignment operator
\end{itemize}
    \end{minipage}
\end{figure}
\end{frame}

\begin{frame}[fragile]{Lambda}
\begin{center}
{\Huge \textbf{Lambda}}
\end{center}
\end{frame}

\begin{frame}[fragile]{Lambda}
\begin{lstlisting}[style=customcpp]
// method
int fun(double aArg1, float aArg2){
  return static_cast<int>(aArg1 + aArg2);
}
 // pointer to method
 int (*ptrFun)(double, float) = fun;
 fun(1.2, 3.4f);
 
 // lambda
 int (*ptrFunLambda)(double, float) =
  [](double aArg1, float aArg2)->int { return static_cast<int>(aArg1 + aArg2);};
  ptrFunLambda(1.2, 3.4f);
\end{lstlisting}
\end{frame}


\begin{frame}[fragile]{Lambda structure - passing arguments}
\begin{lstlisting}[style=customcpp]
[passingArguments](ArgType1 arg1, ArgType2 arg1) mutable -> ReturnType { 
	ReturnType value;
	return value;
}
\end{lstlisting}
\textit{[passingArguments]} - Passing variables to lambda while creating lambda. Possible options: \\

\& - all variables passing by reference, \\
= - all variables passing by value (copy constructor is called),\\
\&, var1, var2 - \textit{var1}, \textit{var2} passing by value, other by reference,\\
=, \&var1, \&var2 - \textit{var1}, \textit{var2} passing by reference, other by value.\\
\begin{lstlisting}[style=customcpp]
auto ptrRef = [=, &str2](int aArg1){ return str1.size() + str2.size() + str3.size() + aArg1;};
}
\end{lstlisting}
\end{frame}

\begin{frame}[fragile]{Lambda structure -  mutable}
\begin{lstlisting}[style=customcpp]
int value = 0;
auto ptrRef = [=]() mutable { ++value;};
ptrRef();
\end{lstlisting}
mutable - it is required if \textit{value} is passed by value.
\end{frame}

\begin{frame}[fragile]{Lambda structure -  return type}
\begin{lstlisting}[style=customcpp]
double value = 12.345;
auto lambda1 = [&]() -> int { return value;};
std::cout<<lambda1();
	
double value1 = 12.345;
double value2 = 54.321;
auto lambda2 = [&]() -> decltype(value1 + value2) { return value1 + value2;};
std::cout<<lambda2();
\end{lstlisting}
\end{frame}


\begin{frame}[fragile]{Lambda with STL}

\begin{lstlisting}[style=customcpp]
std::vector<int> v{1, 2, 3, 4};
const auto it = std::find_if(
	v.begin(), 
	v.end(), 
	[](const int& aArg){return aArg % 2 == 0;});
std::cout<<*it;
\end{lstlisting}
\begin{itemize}
\item STL has many methods in algorithm that end with \textit{\_if} to easily support lambdas.
\item It is very likely that lambda will be inlined.
\end{itemize}
\end{frame}

\begin{frame}[fragile]{auto}
\begin{center}
{\Huge \textbf{auto}}
\end{center}
\end{frame}

\begin{frame}[fragile]{auto}
\begin{lstlisting}[style=customcpp]
std::string fun(){
	return std::string{"as"};
}
std::string& funRef(){
	static std::tring str{"abc"};
	return str;
}
/* ... */
auto var1 = 1;         // <-- var1
auto var2 = fun();     // <-- var2
auto var3 = funRef();  // <-- var3
	
int intValue = 0;
int& intRef = intValue;
auto var4 = intRef;     // <-- var4
\end{lstlisting}
\end{frame}

\begin{frame}[fragile]{auto}
\begin{lstlisting}[style=customcpp]
std::string fun(){
	return std::string{"as"};
}
std::string& funRef(){
	static std::tring str{"abc"};
	return str;
}
/* ... */
auto var1 = 1;         // <-- int
auto var2 = fun();     // <-- std::string
auto var3 = funRef();  // <-- std::string
	
int intValue = 0;
int& intRef = intValue;
auto var4 = intRef;     // <-- int
\end{lstlisting}
\end{frame}

\begin{frame}[fragile]{auto}
\begin{lstlisting}[style=customcpp]
int intValue = 10;
auto var5 = std::move(intValue); // <-- var5
auto& var6 = std::move(intValue); // <-- var6
auto&& var7 = std::move(intValue); // <-- var7

const int intValueC = 0;
auto var8 = intValueC;     // <-- var8
auto& var9 = intValueC; // <-- var9
const auto& var10 = intValueC; // <-- var10

auto var11 = {1, 2, 3}; // <-- var11   
auto var12 = {1}; // <-- var12
auto var13{1}; // <-- var13
auto var14{1, 2};  // <-- var14
\end{lstlisting}
\end{frame}

\begin{frame}[fragile]{auto}
\begin{lstlisting}[style=customcpp]
int intValue = 10;
auto var5 = std::move(intValue); // <-- int
auto& var6 = std::move(intValue); // <-- int&
auto&& var7 = std::move(intValue); // <-- int&&

const int intValueC = 0;
auto var8 = intValueC;     // <-- int
auto& var9 = intValueC; // <-- const int&
const auto& var10 = intValue; // <-- const int&

auto var11 = {1, 2, 3}; // <-- std::initializer_list
auto var12 = {1}; // <-- std::initializer_list, C++17: int
auto var13{1}; // <-- int
auto var14{1, 2};  // <-- Compilation Error
\end{lstlisting}
\end{frame}

\begin{frame}[fragile]{auto}

{\Huge \textit{auto} is \textbf{never} deduced as reference}
\end{frame}

\begin{frame}[fragile]{Variadic templates}
\begin{center}
{\Huge \textbf{Variadic templates}}
\end{center}
\end{frame}

\begin{frame}[fragile]{Variadic templates}
\begin{lstlisting}[style=customcpp]
template<typename T>
void write(const T& aArg){
	std::cout<<aArg<<std::endl;
}
template<typename T1, typename ...T2>
void write(const T1& aArg, const T2& ...aArgs){
	std::cout<<aArg<<std::endl;
	write(aArgs...);
}
template<typename ...T>
void fun(const T&...aArg){
	std::cout<<"Arguments: "<<sizeof...(T)<<std::endl;
	write(aArg...);
}
fun(1, 2, 3, 4);
\end{lstlisting}
\end{frame}

\begin{frame}[fragile]{Enums}
\begin{center}
{\Huge \textbf{Enums}}
\end{center}
\end{frame}

\begin{frame}[fragile]{Enums}
\begin{lstlisting}[style=customcpp]
enum class MyEnum : char { VALUE1, VALUE2};

MyEnum var1 = MyEnum::VALUE2;
MyEnum var2 = static_cast<MyEnum>(1);

std::cout<<static_cast<int>(var1)<<std::endl;
std::cout<<static_cast<int>(var2)<<std::endl;

std::cout<<sizeof(var1);
\end{lstlisting}
\end{frame}

\begin{frame}[fragile]{Suffix}
\begin{center}
{\Huge \textbf{Suffix}}
\end{center}
\end{frame}

\begin{frame}[fragile]{Suffix}
\begin{lstlisting}[style=customcpp]
constexpr long double operator"" _deg (long double deg) {
    return deg * 3.141592 / 180;
}

long double angle = 90_deg;

constexpr std::chrono::minutes operator"" _min(unsigned long long m) {
    return std::chrono::minutes(m);
}

std::chrono::minutes var = 20_min;
\end{lstlisting}
C++14 has several predefined suffixes: min, hours, seconds, etc.
\end{frame}

\begin{frame}[fragile]{Const expressions}
\begin{center}
{\Huge \textbf{Const expressions}}
\end{center}
\end{frame}

\begin{frame}[fragile]{Const expressions}
\begin{lstlisting}[style=customcpp]
struct Class {
	constexpr Class(int aArg1, int aArg2)  : iVar1{aArg1}, iVar2{aArg2}{}
	int iVar1;
	int iVar2;
};
constexpr int fun(const int aArg){
	return 5 + aArg;
}
/* ... */
	constexpr int var1 = 2;
	constexpr int var2 = fun(4);
	constexpr Class cl{1, 2};
\end{lstlisting}
\end{frame}

\begin{frame}[fragile]{Const expressions}
\begin{lstlisting}[style=customcpp]
constexpr int fun1(int aArg){
	return aArg < 0 ? -1 : 1;
}
constexpr int ret = fun1(5);
\end{lstlisting}
C++14:
\begin{lstlisting}[style=customcpp]
constexpr int fun2(int aArg){
	if (aArg < 0)
		return -1;
	else
		return 1;
}
constexpr int ret = fun2(5);
\end{lstlisting}
\end{frame}

\begin{frame}[fragile]{Const expressions}
All methods in header flie.
\begin{lstlisting}[style=customcpp]
struct Class {
	constexpr Class() {}
	constexpr int fun(int aArg) const{
		return iValue + aArg;
	}
	int iValue = 0;
};
\end{lstlisting}
In source file:
\begin{lstlisting}[style=customcpp]
constexpr Class obj;
obj.fun(10);
\end{lstlisting}
\end{frame}

\begin{frame}[fragile]{Initializer list}
\begin{center}
{\Huge \textbf{Initializer list}}
\end{center}
\end{frame}

\begin{frame}[fragile]{Initializer list}
\begin{lstlisting}[style=customcpp]
struct Class {
	Class(std::initializer_list<int> aArg) : iVar(aArg.begin(), aArg.end()){}

	std::vector<int> iVar;
};
	Class var{1, 2, 3, 4};
  
	std::vector<int> var2{1, 2, 3, 4, 5, 6, 7};
\end{lstlisting}
\end{frame}

\begin{frame}[fragile]{Attributes}
\begin{center}
{\Huge \textbf{Attributes}}
\end{center}
\end{frame}

\begin{frame}[fragile]{Attributes: \textit{final}, \textit{override}, \textit{default}, \textit{delete}, }
\begin{lstlisting}[style=customcpp]
struct FinalClass final {
	FinalClass(const FinalClass&) = delete;
	virtual ~FinalClass() = default;
	virtual void overrideFun() override;
	virtual void finalFun() final;
};

template<typename T>
void fun(T& aArg){}

template<>
void fun(int&) = delete;
\end{lstlisting}
\end{frame}

\begin{frame}[fragile]{Foreach}
\begin{center}
{\Huge \textbf{Foreach}}
\end{center}
\end{frame}

\begin{frame}[fragile]{Foreach}
\begin{lstlisting}[style=customcpp]
std::vector<int> var{1, 2, 3};
for (const auto& i : var) {
}
\end{lstlisting}
Foreach requires methods \textit{begin} and \textit{end}.
\end{frame}

\begin{frame}[fragile]{noexcept}
\begin{lstlisting}[style=customcpp]
void fun() noexcept {}
\end{lstlisting}
\textit{throw()} - possible that stack unwinding \\
\textit{noexcept} - now stack unwiding - faster code \\
\end{frame}

\begin{frame}[fragile]{Brackets \{ \}}
\begin{center}
{\Huge \textbf{Brackets \{ \}}}
\end{center}
\end{frame}

\begin{frame}[fragile]{Brackets \{ \}}
\begin{lstlisting}[style=customcpp]
struct Class {
   Class(int aArg){}
};
	Class cl{1.23}; // It will not compile
	Class cl{static_cast<int>(1.23)}; // this is ok
\end{lstlisting}
There is no automatic conversion.
\end{frame}

\begin{frame}[fragile]{Constructor delegate}
\begin{center}
{\Huge \textbf{Constructor delegate}}
\end{center}
\end{frame}

\begin{frame}[fragile]{Calling constructor from constructor}
\begin{lstlisting}[style=customcpp]
struct Class {
	Class() : Class(0){}

	Class(int aArg){}
};
\end{lstlisting}
It is similar to Java.
\end{frame}

\begin{frame}[fragile]{nullptr}
\begin{center}
{\Huge \textbf{nullptr}}
\end{center}
\end{frame}

\begin{frame}[fragile]{nullptr}
\begin{lstlisting}[style=customcpp]
void fun(char* aArg){
	std::cout<<"char*\n";
}

void fun(int aArg){
	std::cout<<"int\n";
}
	fun(0); // int
	fun(NULL); // it will not compile
	fun(nullptr); // char*
\end{lstlisting}
\end{frame}

\begin{frame}[fragile]{STL in C++11} 
\begin{center}
{\Huge \textbf{STL in C++11}}
\end{center}
\end{frame}

\begin{frame}[fragile]{Thread}
\begin{center}
{\Huge \textbf{Thread}}
\end{center}
\end{frame}

\begin{frame}[fragile]{Thread - mutex} 
\begin{lstlisting}[style=customcpp]
struct Counter {
    std::mutex mutex; // <-- mutex
    int value = 0;

    void increment(){
        std::lock_guard<std::mutex> guard(mutex); // <-- lock
        ++value;
    }
};

\end{lstlisting}
\end{frame}

\begin{frame}[fragile]{Thread - mutex II} 
\begin{lstlisting}[style=customcpp]
 for(int i = 0; i < 5; ++i){
	        threads.push_back(std::thread([&counter](){
	            for(int i = 0; i < 100; ++i){
	                counter.increment();
	            }
	        }));
	    }

	    for(auto& thread : threads){
	        thread.join(); // <-- wait
	    }
\end{lstlisting}
\end{frame}

\begin{frame}[fragile]{Thread - async} 
\begin{lstlisting}[style=customcpp]
#include <thread>
int fun(){
	std::cout<<"Thread id: "<<std::this_thread::get_id()<<std::endl;
	int i = 0;
	for (; i < 100; ++i){
		std::this_thread::sleep_for(std::chrono::nanoseconds{100000000});
	}
	return i;
}
\end{lstlisting}
\end{frame}

\begin{frame}[fragile]{Thread - async II} 
\begin{lstlisting}[style=customcpp]
#include <future>
int main() {
	std::cout<<"Main Thread id: "<<std::this_thread::get_id()<<std::endl;
	std::vector<std::future<int>> v;
	for (int i = 0; i < 2; ++i) {
		std::future<int> ret = std::async(std::launch::async, fun);
		v.push_back(std::move(ret));
	}
	std::cout<<"wait"<<std::endl;
	for (std::future<int>& fut : v) {
		int value = fut.get();
		std::cout<<value<<std::endl;
	}
	return 0;
}
\end{lstlisting}
\end{frame}

\begin{frame}[fragile]{Thread - async II} 
\begin{lstlisting}[style=customcpp]
std::future<int> ret = std::async(std::launch::async, fun);
\end{lstlisting}
\begin{itemize}
\item std::launch::async - new thread
\item std::launch::deferred - caller thread
\end{itemize}

Destructor of \textit{std::future<T>} waits for execution of async call.\\
C++14: waits if all are true
\begin{itemize}
\item created from std::async
\item shared object is not ready
\item the last reference of shared object
\end{itemize}
\end{frame}

\begin{frame}[fragile]{Smart pointers}
\begin{center}
{\Huge \textbf{Smart pointers}}
\end{center}
\end{frame}

\begin{frame}[fragile]{Smart pointers}
\textbf{DON'T USE RAW POINTERS}
\begin{lstlisting}[style=customcpp]
#include <memory>

std::shared_ptr<int> ptr1 = std::make_shared<int>(2);

std::shared_ptr<int> ptr2(new int[size], [](int* aArg){ delete[] aArg;});

std::unique_ptr<int> ptr3(new int()); 

std::unique_ptr<int> ptr4 = std::make_unique<int>(); // C++14
\end{lstlisting}
\end{frame}


\begin{frame}[fragile]{Smart pointers}
\textbf{DON'T USE RAW POINTERS}
\begin{lstlisting}[style=customcpp]
#include <memory>

struct Base {
	virtual ~Base() = default;
};
struct Derivative : public Base {};

std::shared_ptr<Base> ptrBase = 
	std::make_shared<Derivative>();
std::shared_ptr<Derivative> ptrDerivative = 
	std::dynamic_pointer_cast<Derivative>(ptrBase);
\end{lstlisting}
\end{frame}


\begin{frame}[fragile]{Smart pointers}
\textbf{DON'T USE RAW POINTERS}
\begin{lstlisting}[style=customcpp]
#include <memory>

struct Class {};

std::shared_ptr<Class> ptr = std::make_shared<Class>();
std::weak_ptr<Class> weakPtr = ptr;

std::shared_ptr<Class> sharedPtrFrom = weakPtr.lock();
/* using sharedPtrFrom */
\end{lstlisting}
\end{frame}

\begin{frame}[fragile]{tuple}
\begin{center}
{\Huge \textbf{tuple}}
\end{center}
\end{frame}

\begin{frame}[fragile]{std::tuple}
\begin{lstlisting}[style=customcpp]
#include <tuple>
std::tuple<int, double> fun(){
	return std::make_tuple(1, 2.3);
}
// ...
	auto ret = fun();
	int intValue = std::get<0>(ret);
	double doubleValue = std::get<1>(ret);

\end{lstlisting}
\end{frame}

\begin{frame}[fragile]{std::tie}
\begin{lstlisting}[style=customcpp]
#include <tuple>
std::tuple<int, double> fun(){
	return std::make_tuple(1, 2.3);
}
// ...
	int intValue;
	double doubleValue;
	std::tie(intValue, doubleValue) = fun();
	std::cout<<intValue<<" "<<doubleValue<<std::endl;
	
	// C++ 17 (now only Clang):
	auto [intValue, doubleValue] = fun();
\end{lstlisting}
\end{frame}

\begin{frame}[fragile]{array}
\begin{center}
{\Huge \textbf{array}}
\end{center}
\end{frame}

\begin{frame}[fragile]{std::array}
\textbf{DON'T USE RAW ARRAYS}
\begin{lstlisting}[style=customcpp]
#include <array>
std::array<int, 5> arr1 = {1, 2, 3};

// C++14 only one {}
std::array<int, 5> arr2{{1, 2, 3}};

std::cout<<"size: "<<arr1.size()<<std::endl;
for (auto& it : arr1){
	std::cout<<it<<std::endl;
}
\end{lstlisting}
\end{frame}

\begin{frame}[fragile]{bind}
\begin{center}
{\Huge \textbf{bind}}
\end{center}
\end{frame}

\begin{frame}[fragile]{std::bind}
\begin{lstlisting}[style=customcpp]
#include <functional>
void fun1(){
	std::cout<<"void fun1()\n";
}
void fun2(int aArg){
	std::cout<<"void fun2(int aArg)\n";
}
// ...
int intValue = 1;
std::function<void()> functor1a = fun1;
std::function<void()> functor1 = std::bind(fun1);
functor1();
// with one argument
std::function<void()> functor2 = std::bind(fun2, intValue);
functor2();
\end{lstlisting}
\end{frame}

\begin{frame}[fragile]{std::bind}
\begin{lstlisting}[style=customcpp]
#include <functional>
void fun3(int& aArg, double* aPtr){
	std::cout<<"void fun3(int& aArg, double* aPtr)\n";
}
// ...
std::function<void()> functor3 = std::bind(fun3, intValue, nullptr);
functor3();

using namespace std::placeholders;
std::function<void(int&)> functor4 = std::bind(fun3, _1, nullptr);
functor4(intValue);

std::function<void(double*, int&)> functor5 = std::bind(fun3, _2, _1);
functor5(nullptr, intValue);
\end{lstlisting}
\end{frame}

\begin{frame}[fragile]{std::bind}
\begin{lstlisting}[style=customcpp]
#include <functional>
struct Class{
	void method(int aArg)  {
		std::cout<<"void Class::method(int aArg)\n";
	}
};
// ...
Class obj;
// class method
std::function<void( Class*, int)> functor6 = &Class::method;
functor6(&obj, intValue);
// class method with placeholder
std::function<void(int)> functor7 = std::bind(&Class::method, &obj, _1);
functor7(intValue);
\end{lstlisting}
\end{frame}

\begin{frame}[fragile]{Type traits}
\begin{center}
{\Huge \textbf{Type traits}}
\end{center}
\end{frame}

\begin{frame}[fragile]{Type traits}
\begin{lstlisting}[style=customcpp]
#include <type_traits>
template<typename T>
void traits(){
	cout<<is_array<T>::value<<"\n";
	cout<<is_class<T>::value<<"\n";
	cout<<is_function<T>::value<<"\n";
	cout<<is_pointer<T>::value<<"\n";
	cout<<is_polymorphic<T>::value<<"\n";
}
//...
traits<int>();
traits<std::string>();
\end{lstlisting}
\end{frame}

\begin{frame}[fragile]{Type traits - std::enable\_if}
\begin{lstlisting}[style=customcpp]
#include <type_traits>
template<class T>
typename std::enable_if<std::is_integral<T>::value, bool>::type
  is_odd (T i) {
	return i % 2;
}
//...
int intValue = 0;
is_odd(intValue);
std::string strValue{"abc"};
is_odd(strValue); // Error
\end{lstlisting}
\end{frame}

\begin{frame}[fragile]{Type traits - std::enable\_if}
\begin{lstlisting}[style=customcpp]
#include <type_traits>
template<class T,
         class = typename std::enable_if<std::is_integral<T>::value>::type>
bool is_even (T i) {
	return !bool(i % 2);
}
//...
int intValue = 0;
is_even(intValue);
std::string strValue{"abc"};
is_even(strValue); // Error
\end{lstlisting}
\end{frame}

\begin{frame}[fragile]{ratio}
\begin{center}
{\Huge \textbf{ratio}}
\end{center}
\end{frame}

\begin{frame}[fragile]{std::ratio}
\begin{lstlisting}[style=customcpp]
#include <ratio>
std::ratio<1,3> one_third;
std::ratio<2,4> two_fourths;

std::cout << "one_third= " << std::ratio<1,3>::num
	<< "/" << std::ratio<1,3>::den << std::endl;
std::cout << "two_fourths= " << std::ratio<2,4>::num
	<< "/" << std::ratio<2,4>::den << std::endl;

std::ratio_add<std::ratio<1,3>, std::ratio<2,4>> sum;
std::cout << "sum= " <<
	std::ratio_add<std::ratio<1,3>, std::ratio<2,4>>::num
	<< "/" <<
	std::ratio_add<std::ratio<1,3>, std::ratio<2,4>>::den;

\end{lstlisting}
\end{frame}

\begin{frame}[fragile]{chrono}
\begin{center}
{\Huge \textbf{chrono}}
\end{center}
\end{frame}

\begin{frame}[fragile]{std::chrono}
\begin{lstlisting}[style=customcpp]
#include <chrono>
std::chrono::time_point<std::chrono::system_clock> start, end;
start = std::chrono::system_clock::now();
std::cout<<"write sth\n";
end = std::chrono::system_clock::now();
std::chrono::nanoseconds diff = end - start;
std::cout<<diff.count()<<"ns\n";

std::time_t end_time = std::chrono::system_clock::to_time_t(end);
std::cout << "finished computation at " << std::ctime(&end_time);
// write sth
// 25000ns
// finished computation at Sun Dec 18 17:06:17 2016
\end{lstlisting}
\end{frame}

\begin{frame}[fragile]{C++14}
\begin{center}
{\Huge \textbf{C++14}}
\end{center}
\end{frame}

\begin{frame}[fragile]{C++14}
Binary literal \& separator:
\begin{lstlisting}[style=customcpp]
int binary = 0b101; // = 5
int bigNumber = 1'000'000; // ' - separator
\end{lstlisting}
Automatic return type decuction:
\begin{lstlisting}[style=customcpp]
auto fun(bool aArg){
	if (aArg)
		return 1;
	else
		return 0;
}
\end{lstlisting}
\end{frame}

\begin{frame}[fragile]{C++14}
Generic lambda:
\begin{lstlisting}[style=customcpp]
auto lambda = [](auto x, auto y) {return x + y;};
std::cout<<lambda(1, 2.3);
\end{lstlisting}
Lambda capture expressions:
\begin{lstlisting}[style=customcpp]
int var = 1;
auto lambda = [inside = std::move(var)](auto x, auto y) {return x + y + inside;};
std::cout<<lambda(1, 2.3);
\end{lstlisting}
\end{frame}

\end{document}

